\documentclass{article}

\usepackage{amsmath}
\usepackage{pifont}
\usepackage{listings}
\usepackage{amssymb}
\usepackage{graphicx}
\usepackage[margin=1.0in]{geometry}
\usepackage[table]{xcolor}
\usepackage{multicol}

\title{COEN 20 Homework \#7}
%Ch 8 All questions 
\date{13 May 2013}
\author{Rick Sullivan}

\begin{document}
\noindent Rick Sullivan     \\
          Professor Lewis   \\
          COEN 20           \\
          20 May 2013        \\
                            \\
\centerline {Homework \#7}    \\
                            \\
\begin{enumerate}
    \item
        c) DMA
    \item
        c) DMA
    \item
        c) DMA
    \item
        c) Maximum memory bandwidth
    \item
        \begin{enumerate}
            \item   2 clock cycles
            \item   2 clock cycles
            \item   1 clock cycle
            \item   5 clock cycles
            \item   4 clock cycles
        \end{enumerate}
    \item
        \begin{enumerate}
            \item   1 byte from memory
            \item   4 bytes from memory
            \item   0 bytes
            \item   16 bytes to memory
            \item   4 bytes to memory
        \end{enumerate}
    \item a, b, c, and d
    \item a
    \item a, b, and c
    \item a) R0, c) R12, d) LR, and e) PSW
    \item c) The NVIC
    \item a) A hard-coded constant built into the CPU
    \item False
    \item b) Tail-chaining
    \item c) Late arrival processing
    \item b) The starting address of the vector table is stored in the vector table offset register
    \item 
        \begin{enumerate}
            \item If the second interrupt is of \textbf{lower} priority, then only after the ISR returns.
            \item Immediately if the second interrupt is of \textbf{higher} priority.
        \end{enumerate}
    \item Registers R0-R3, R12, LR, Return Address, and PSR. 
    \item Interrupt-driven I/O is slowed by the overhead of saving and restoring the machine state.
    \item 
        \begin{enumerate}
            \item This case would be the best case latency of 340nsec.
            \item The latency depends on whether the new ISR has a higher or lower priority than the currently executing ISRs.
            \item DMA has a latency of approximately 60nsec.
            \item The latency of the polled waiting loop is unpredictable.
        \end{enumerate}
    \item 
\begin{lstlisting}
void move(uint8_t *from, uint8_t *to)
{
    for(int byte = 0; byte < 1000; byte++)
    { 
        *to = *from;
        to++; 
        from++;
    }
}
   
In assembly:

        ; R0 has &from
        ; R1 has &to
move:

        LDR R3, =0         ; byte counter
Start:  CMP R3, #250       ; It's faster to do 4 bytes per loop,
        BHE Return         ;    so we only need to loop 250 times
        LDR R2, [R0], #1   ; Get next byte and update pointer
        STR R2, [R1], #1   ; Store byte and update pointer
        ADD R3, R3, #1     ; Increment counter.
        B   Start
Return: BX  LR
\end{lstlisting}

    Each loop should take approximately 8 clock cycles, and it will execute
250 times. Therefore, it will take 2,000 clock cycles (+ some negligible
cycles outside of the loop). At 20nsec per cycle, this should take about
40,000 nsec.
\end{enumerate}
\end{document}

