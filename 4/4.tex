\documentclass{article}

\usepackage{amsmath}
\usepackage{pifont}
\usepackage{listings}
\usepackage{amssymb}
\usepackage{graphicx}
\usepackage[margin=1.0in]{geometry}
\usepackage[table]{xcolor}

\title{COEN 20 Homework \#4}
%Ch 5 Problems 1-9, 12-19 
\date{29 April 2013}
\author{Rick Sullivan}

\begin{document}
\noindent Rick Sullivan     \\
          Professor Lewis   \\
          COEN 20 - Embedded Systems    \\
          29 April 2013     \\
                            \\
\centerline{Homework \#4}   \\
                            
\begin{enumerate}
	\item   a) I/O-MEM-CPU
	\item   b) CPU
	\item   \begin{enumerate}
                \item   \begin{align*}
                        256 kB &= 256 * 1024\ bytes          \\
                        lg(256*1024) &= 18\ address\ lines.    
                        \end{align*}
                \item   \begin{align*} 
                        2 MB &= 2 * 1048576\ bytes             \\
                        lg(2 * 1048576) &= 21\ address\ lines.
                        \end{align*}
                \item   \begin{align*}
                        4 GB = 4 * 1024 MB &= 4 * 1024 * 1048576\ bytes  \\
                        lg(4 * 1024 * 1048576) &= 32\ address\ lines.
                        \end{align*}
            \end{enumerate}
	\item   \begin{enumerate}
                \item   Data is stored in the \underline{MEMORY}.
                \item   Programs are stored in the \underline{MEMORY}.
                \item   To be executed, an instruction is loaded into the \underline{IR}.
                \item   The address of an instruction to be executed is held in \underline{PC}.
                \item   Instructions are retried during the \underline{FETCH PHASE}.
                \item   Instructions are interpreted during the \underline{EXECUTE PHASE}.
                \item   Arithmetic operations occur in the \underline{ALU}.
                \item   Temporary intermediate results are held in the \underline{ACC}.
            \end{enumerate}
	\item   The fetch phase is preceded by the \underline{EXECUTE} phase.
	\item   The fetch phase is followed by the \underline{EXECUTE} phase.
	\item   The instruction register (IR) is loaded during the \underline{FETCH} phase.
	\item   Computation performed by an instruction occurs during the \underline{EXECUTE} phase.
	\item   In the ARM architecture, there are:
        \begin{enumerate}
            \item 2 bytes in a half word.
            \item 8 bytes in a double word.
        \end{enumerate}
    \pagebreak
    \setcounter{enumi}{11}
	\item   
        \begin{enumerate}
            \item
                \begin{tabular}{ p{2cm} | p{2cm} | p{2cm} | p{2cm} }   
                    \hline
                    Address 111 & Address 110 & Address 109 & Address 108   \\ \hline
                    Address 107 & Address 106 & \cellcolor{gray}Address 105 & \cellcolor{gray}Address 104   \\ \hline
                    \cellcolor{gray}Address 103 & \cellcolor{gray}Address 102 & Address 101 & Address 100   \\ \hline
                    Address 99  & Address 98  & Address 97  & Address 96    \\ \hline
                \end{tabular}
             \item  Address 105.
             \item  2 memory cycles.
             \item  1 memory cycle.
         \end{enumerate}
     \item  b) The most significant byte.
     \item  Address 100.
     \item  \begin{enumerate}
                \item   Address N + 3.
                \item   Address N + 2. 
            \end{enumerate}
     \item  b) Big-endian numbering.
     \item  a) From least to most significant bit, starting at 0.
     \item  \begin{enumerate}
                \item   101.
                \item   102.
                \item   104.
                \item   108.
            \end{enumerate}
     \item  b)PC, c)LR, and d)PSW.
\end{enumerate}

\end{document}

