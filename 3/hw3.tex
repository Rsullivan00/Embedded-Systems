\documentclass{article}

\usepackage{amsmath}
\usepackage{amssymb}
\usepackage{graphicx}

\title{COEN 20 Homework \#3}
%Ch 4 Q's 21, 22, 24, 25
	%\subtitle{COEN 20}
\date{22 April 2013}
\author{Rick Sullivan}

\begin{document}
\maketitle

\begin{enumerate}
	\item[21.]
		\begin{enumerate}
			\item 	\begin{tabbing}
				\texttt{int bit = 16;} 	\\
				\texttt{int temp = x;} 	\\
				\texttt{while(bit < 24)$\{$ } \\
				\hspace{4ex}	\texttt{temp \^\ = (1 << bit);}	\\
				\hspace{4ex}	\texttt{if(temp > x) break;}	\\
				\hspace{4ex}	\texttt{bit++;}		\\
				\texttt{$\}$}		\\
				\texttt{x = temp;}	\\
				\end{tabbing}

			\item 	\texttt{((short*)\&x)[3] += 1;}
		\end{enumerate}

	\item[22.] \texttt{A63F}

	\item[24.] 
		\begin{tabular}{ p{4cm} | l | l | l | l}
			\hline
			% This first row is a little messy
			\textit{Characteristic} & \textit{Bitwise Operators} & 
			\textit{Structure Bit Fields} & \textit{Variant Acess} & 
			\textit{Unions} \\ \hline
			% End of first row
			Messy syntax? &	YES & NO & NO & NO\\ \hline 
			Ambiguous semantics? & NO & YES & YES & YES\\ \hline
			Bit-field assignment restrictions? & NO & YES & YES & YES\\ \hline
			More efficient code generation? & YES & YES & YES & NO \\ \hline

		\end{tabular}
	
	\item[25.]
		\begin{enumerate}
			\item \texttt{ short value = (packed >> 8); }

			\item \texttt{ short value = packed.c[1]; }
		\end{enumerate}

\end{enumerate}

\end{document}
